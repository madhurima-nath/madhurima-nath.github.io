

\subsection{Post-doctoral Research Assistant $\|$ Virginia Tech, USA $\|$ {\textit{Supervisor}}: Prof. Stephen Eubank \hfill Feb 2019 -- Dec 2019}
\begin{zitemize}
\item  Developed hybrid computational approach combining Monte Carlo simulations with weak- and strong-coupling perturbative expansions to improve estimation of Moore-Shannon network reliability on graphs, addressing an NP-hard computational problem. [{\ref{arxiv-paper}}]

\item Extended weak- and strong-coupling perturbative methods from statistical physics to heterogeneous satisfiability problems, developing novel approach for constructing tight upper and lower bounds on approximation error for mostly monotonic probabilistic satisfiability problems. [{\ref{arxiv-paper}}]

\item Implemented statistical approaches for community detection in large-scale weighted directed networks, achieving significant improvements over traditional methods.
[{\ref{com-det}}]

\item Applied computational methods to analyse international trade network dynamics from the United Nations (UN) Comtrade database, to identify crucial communities for preventing global pestilence distribution. [{\ref{com-det}}]
\end{zitemize}

\subsection{Graduate Research Assistant $\|$ Virginia Tech, USA $\|$ {\textit{Supervisor}}: Prof. Stephen Eubank \hfill May 2014 -- Dec 2018}
\begin{zitemize}
\item Applied Moore-Shannon network reliability formalism (traditionally used for electronic circuit analysis) to biological and social network systems, establishing theoretical framework connecting electrical engineering concepts to network epidemiology and complex systems analysis. {\href{https://github.com/NSSAC/reliability}{\underline{
\textcolor{black}{GitHub repo}
}}} [{\ref{comp-theory}, \ref{epi-paper}}]

\item Developed computational approaches to evaluate Moore-Shannon network reliability formalism using Bernstein basis functions as polynomial basis set, providing convergence guarantees for sequential design and model selection in discrete finite systems.
{\href{https://github.com/NSSAC/reliability}{\underline{
\textcolor{black}{GitHub repo}
}}} [{\ref{arxiv-paper}}]

\item Applied Moore-Shannon network reliability to predict final global states of graph dynamical systems, analysing how interactions between individual node states and their connections determine final outcomes in practical applications.[{\ref{arxiv-paper}, \ref{comp-theory}}]

\item Showed that synthetic network models generated using Exponential Random Graph Models cannot reliably predict epidemic outcomes on empirical contact networks. 
Analysis of National Longitudinal Study of Adolescent to Adult Health (Add Health) dataset demonstrated that models matching local network statistics can overestimate infection numbers by approximately 50\%. Applied Birnbaum importance measures to quantify individual edge contributions to epidemic potential, enabling identification of critical connections for targeted. [{\ref{epi-paper}}]

\item Identified vulnerabilities within global food trade networks using commodity-specific data from UN Comtrade database, analysing where pest and pathogen contamination could cascade through global supply chains. Achieved approximately 96\% precision in forecasting impact of mitigation strategies under various contagion scenarios. [{\ref{trade}}]

\item Reduced time complexity from $O(n^2)$ to $O(n)$ for estimating energy states in interacting magnetic systems (Ising model) by reformulating the problem through network reliability framework. Established equivalence between network reliability and Ising partition function, enabling transfer of statistical physics computational methods to network science applications. [{\ref{ising}}]

\item  Developed parallel Markov-chain Monte Carlo scheme for estimating joint density of states in Ising model applications, addressing fundamental limitations of naive sampling approaches that yield poor approximations for partition function estimation. [{\ref{ising}}]
\end{zitemize}






%% masters
\subsection{Master's Thesis $\|$ Indian Institute of Technology Delhi, India $\|$ {\textit{Supervisor}}: Prof. Sankalpa Ghosh \hfill Jul 2011 -- May 2012}
\vspace{0.1cm}
Study of Cold Atomic Condensates by Atom-Photon Interactions [{\ref{msc}}] $\|$ {\href{https://github.com/madhurima-nath/madhurima-nath.github.io/blob/main/docs/IIT_Award.pdf} {\textcolor{black}{Best Master of Science Thesis 2011-2012}}}
\begin{zitemize}
\item Developed computational framework in MATLAB for probing quantum many-body states of ultracold atoms in optical lattices using angle-resolved light transmission measurements, demonstrating that confined atoms act as quantum diffraction gratings.
\item Calculated dispersive shifts in cavity resonance caused by atomic presence, establishing direct proportionality between resonance shift and atom count in illuminated sites for systematic quantum state characterisation.
\item Generated transmission spectra across Mott Insulator and Superfluid phases by systematically varying cavity-lattice geometry, atom number, and illuminated site distribution parameters.
\item Established non-destructive measurement approach enabling experimental exploration of Fock-space structure in few-body correlated quantum systems through optical cavity transmission analysis.
\item Visualisation selected for American Physical Society (APS) Physical Review A 
{\href{https://journals.aps.org/pra/kaleidoscope/pra/85/6/063606}{\underline{\textcolor{black}{Kaleidoscope (June 2012)}}}} based on aesthetic quality of graphics.
\end{zitemize}