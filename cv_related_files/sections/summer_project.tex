

\subsection{Quantum Information Conservation: The No-Hiding Theorem $\|$
{\href{https://github.com/madhurima-nath/madhurima-nath.github.io/blob/main/docs/hri_completion_letter2012.jpg}{\underline{\textcolor{black}{Completion letter}}}} \hfill Jun 2012 -- Jul 2012}
\\
Harish-Chandra Research Institute, India $\|$
{\textit{Supervisor}}: Prof. Arun Kumar Pati
\begin{zitemize}
\item Selected for prestigious Visiting Students Programme (VSP) in Physics at premier research institute funded by Department of Atomic Energy, Government of India, to conduct research in quantum information theory.

\item Studied fundamental conservation laws governing information in quantum systems through theoretical analysis and numerical verification of the No-Hiding Theorem, establishing that quantum information cannot be created or destroyed but only redistributed between system and environment.
\item Analysed perfect hiding scenarios where quantum information transfers completely to ancilla (environment) states without entanglement, demonstrating information conservation in the absence of quantum correlations.
\item Quantified how information about input states distributes between system and environment in imperfect hiding cases using correlation measures implemented in MATLAB.
\end{zitemize}


\subsection{Renormalisation Group Study of Li\'enard Systems $\|$
{\href{https://github.com/madhurima-nath/madhurima-nath.github.io/blob/main/docs/ju_completion_letter2011.pdf}{\underline{\textcolor{black}{Completion letter}}}} \hfill May 2011 -- Jul 2011}
\\
Jadavpur University, India $\|$
{\textit{Supervisor}}: Dr. Dhruba Banerjee
\begin{zitemize}
\item Applied renormalisation group methods to analyse limit cycle behaviour in nonlinear dynamical systems, working within the framework of Hilbert's sixteenth problem on limit cycles in polynomial differential equations.
\item Derived amplitude equations for generalised Li\'enard systems using perturbative renormalisation group techniques, systematically incorporating higher-order damping effects through recursive expansions.
\item Determined limit cycle existence and stability by analysing fixed points and eigenvalues of the amplitude equations.
\item Extended the perturbative analysis to second order, examining the structure and convergence properties of the analytical approach for higher-order approximations.
\item Validated theoretical predictions through computational analysis in Mathematica, generating phase portraits and limit cycle trajectories across parameter space.

\end{zitemize}